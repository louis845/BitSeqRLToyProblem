\documentclass{article}


% if you need to pass options to natbib, use, e.g.:
%     \PassOptionsToPackage{numbers, compress}{natbib}
% before loading neurips_2024


% ready for submission
\usepackage[preprint]{neurips_2024}


% to compile a preprint version, e.g., for submission to arXiv, add add the
% [preprint] option:
%     \usepackage[preprint]{neurips_2024}


% to compile a camera-ready version, add the [final] option, e.g.:
%     \usepackage[final]{neurips_2024}


% to avoid loading the natbib package, add option nonatbib:
%    \usepackage[nonatbib]{neurips_2024}


\usepackage[utf8]{inputenc} % allow utf-8 input
\usepackage[T1]{fontenc}    % use 8-bit T1 fonts
\usepackage{hyperref}       % hyperlinks
\usepackage{url}            % simple URL typesetting
\usepackage{booktabs}       % professional-quality tables
\usepackage{amsfonts}       % blackboard math symbols
\usepackage{nicefrac}       % compact symbols for 1/2, etc.
\usepackage{microtype}      % microtypography
\usepackage{xcolor}         % colors


\title{Documentation of Bit Sequence mini RL project}


% The \author macro works with any number of authors. There are two commands
% used to separate the names and addresses of multiple authors: \And and \AND.
%
% Using \And between authors leaves it to LaTeX to determine where to break the
% lines. Using \AND forces a line break at that point. So, if LaTeX puts 3 of 4
% authors names on the first line, and the last on the second line, try using
% \AND instead of \And before the third author name.


\author{%
  Chau Yu Hei\thanks{Currently not a student yet :(} \\
  HKUST\\
  20644747\\
  \texttt{yhchau@connect.ust.hk} \\
  % examples of more authors
  % \And
  % Coauthor \\
  % Affiliation \\
  % Address \\
  % \texttt{email} \\
  % \AND
  % Coauthor \\
  % Affiliation \\
  % Address \\
  % \texttt{email} \\
  % \And
  % Coauthor \\
  % Affiliation \\
  % Address \\
  % \texttt{email} \\
  % \And
  % Coauthor \\
  % Affiliation \\
  % Address \\
  % \texttt{email} \\
}


\begin{document}


\maketitle


\begin{abstract}
GFlowNets provide a new paradigm for finetuning LLMs with respect to some energy function. However, a common problem is that strings with high probability are often very rare compared to the strings with low probability. How do we efficiently tune LLMs using the GFlowNet algorithm so that it is able to learn from sparse rewards? In this mini-project, we attempt to tackle the problem of generating a randomly prechosen bit string using old RL methods.
\end{abstract}


\section{Background}
Recent work has shown that GFlowNets are able to generate diverse samples for LLMs, and finetune LLMs according to Bayesian updates (\cite{hu_llm}). However, their experiments are limited to generating text sequences of short length. How do we generate text sequences of larger lengths? It is evident that most strings have low reward due to the large search space of the set of all possible text sequences.

In this mini-project, we attempt to tackle a similar toy problem, that is, to generate a specific bit sequence in the set of all bit sequences. We first apply a naive DQN method for sequence generation, and show that the method only works for small $n$. Then we attempt to apply Hindsight Experience Replay (\cite{hindsight_openai}) by OpenAI to see whether the method successfully generates the correct bit sequences for longer sequence lengths.

\section{Problem setting}
Here we attempt to use a DQN to generate a predesignated randomly picked target bit sequence. In each setting, an integer $1 \leq n \leq 50$ is fixed, and the state space is $\mathcal{S}_n=\{0,1\}^n$. Before the training process, we pre-pick a sequence $s \in \mathcal{S}_n$ randomly (e.g $00110$ for $\mathcal{S}_5$). During training, the goal of the DQN network is to generate the sequence $s$, where the reward is $1$ for arriving at $s$, and $0$ for arriving at other $\mathcal{S}_n \ni s' \neq s$. 

The intuition is that, for low values of $n$ (maybe $1 \leq n \leq 10$?), a naive DQN is able to learn to arrive at the bit sequence due to exhausting the search space (along with using the parallel processing by GPUs). However, for large values of $n$, this would be infeasible. We will further explore generating the sequence using other methods.


\section{Experiments}
Coming soon ASAP......

\bibliographystyle{plainnat}
\bibliography{citations}
\end{document}